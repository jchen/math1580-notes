%!TEX root = ../notes.tex
\section{February 9, 2022}
\subsection{Asymmetric/Public Key Cryptography}
The premise is that we have \emph{Alice} and \emph{Bob} who are communicating, and \emph{Eve} intercepts \ul{all} communications between them. There is \textbf{no} communication between Alice and Bob ahead of time. A priori, it's not entirely obvious that this is possible\dots

We'll see that this is indeed possible!

\begin{example}
    \ul{Analogy}: Alice and Bob are communicating by writing messages on pieces of paper.

    Symmetric cryptography is having a shared safe, Alice and Bob both have the key/know the combination to, and both can leave messages and retrieve messages.

    \begin{enumerate}
        \item Alice sets up a box with a thin slot with a lock on it. Alice has the key to this lock.
        \item Bob is able to deposit messages into the slot in the box, and Alice can retrieve it using her key.
    \end{enumerate}
\end{example}

Our key is now $k = (k_\mathsf{priv}, k_\mathsf{pub})\in \mathcal{K} = \mathcal{K}_\mathsf{priv}\times \mathcal{K}_\mathsf{pub}$ which consists of a private key and public key.

Our encryption and decryption functions are now
\begin{align*}
    e & : \mathcal{K}_\mathsf{pub} \times \mathcal{M} \to \mathcal{C}  \\
    d & : \mathcal{K}_\mathsf{priv} \times \mathcal{C} \to \mathcal{M}
\end{align*}
\[d(k_\mathsf{priv}, e(k_\mathsf{pub}, m)) = m\]
We want it to be easy to compute $e_{k_\mathsf{pub}}$ and $d_{k_\mathsf{priv}}$, but hard to compute $d_{k_\mathsf{priv}}$ only knowing $k_\mathsf{pub}$.

Something easier to construct, before a full-fledged public key system, is a key exchange:
\subsection{Diffie-Hellman Key Exchange}
\textbf{Q: How can Alice and Bob agree on a secret key over an insecure channel?}

\begin{example}
    \ul{Analogy}: A lockbox that can only be used by one person...and both people have to participate to set it up.
\end{example}

Both parties have to agree on a key and have a line of communication before agreeing on a key. This can only be used if both parties are online at the same time.

We start with a prime $p$ and $g\in(\ZZ/p\ZZ)^\times$ suitably. Alice and bob does the following, all mod $p$:
\begin{center}
    \begin{tabular}{ll}
        \toprule
        Alice              & Bob                 \\ \midrule
        Generates $a$      & Generates $b$       \\
        \quad $\downarrow$ & \quad $\downarrow$  \\
        Computes $g^a$     & Computes $g^b$      \\
        Send $g^a$ to Bob  & Send $g^b$ to Alice \\
        Computes $(g^b)^a$ & Computes $(g^a)^b$  \\
        \bottomrule
    \end{tabular}
\end{center}

Alice and Bob now know $g^{ab}$, which is the secret key. Eve, however, only knows $g^a$ and $g^b$. Alice and Bob can now use this shared secret $g^{ab}$ as a key for symmetric cryptography.

\begin{definition}[The Diffie-Hellman Problem (DHP)]
    Given $g^a, g^b$, calculate $g^{ab}$.
\end{definition}
\begin{remark}
    If we can solve the discrete log problem, we can solve the Diffie-Hellman problem.
\end{remark}

Vice versa? Can one solve DLP given solution to DHP? \emph{This is unknown\footnote{There is no known method.}.}

\subsection{Elgamal Public Key Cryptography}
We again start with $p$ prime and $g\in (\ZZ/p\ZZ)^\times$ suitably. This could be public knowledge, or Alice selects these.

Alice: We have $a$ be Alice's private key, and $A = g^a$ be Alice's public key.

Bob: Has message $m$ he wishes to send. Bob does the following:
\begin{enumerate}
    \item Generate random $k$ (used only once, to send this message).
    \item Compute the following:
          \begin{enumerate}
              \item $c_1 = g^k\mod{p}$
              \item $c_2 = m\cdot A^k \mod{p}$
          \end{enumerate}
    \item Send $c_1$ and $c_2$ to Alice.
\end{enumerate}

Alice:
\[(c_1^a) = A^k \text{ so } c_2\cdot (c_1^a)^{-1}\equiv m\left((g^a)^k\right)\cdot \left((g^a)^k\right)^{-1} \equiv m\]
Basically, they are using Diffie-Hellman key exchange, except $g^a$ is a public key and Bob assumes a secret key, and uses that to encrypt the message and sends it in one go.

\subsubsection{Implementation}
We have the following algorithm for encryption and decryption in Elgamal: 
\begin{lstlisting}[language=Python]
import ext_gcd, pow_mod
from random import randrange
def e(A, m): 
    k = randrange(p)
    return (pow_mod(g, k, p), m * pow_mod(A, k, p))

def d(a, c):
    pow_mod(c[0])
    ...
\end{lstlisting}
\emph{to be continued...}
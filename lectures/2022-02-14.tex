%!TEX root = ../notes.tex
\section{February 14, 2022}
\subsection{Groups \emph{continued}}
\begin{example}
    Some itemize of groups and nongroups:
    \begin{itemize}
        \item $(\ZZ/N\ZZ, +)$: Yes (Abelian).
        \item $(\ZZ/N\ZZ, \times)$: No. $0^{-1}$ does not exist (inverse).
        \item $((\ZZ/n\ZZ)^\times, \times)$: Yes (Abelian).
        \item $(\ZZ\setminus \{0\}, \times)$: No. $2^{-1}$ does not exist (inverse).
        \item $(\ZZ\setminus \{0\}, +)$: No. No identity $e$.
        \item $\left(\{n\times n\text{ matrices} : \det M \neq 0\}, \times\right)$: Yes (not Abelian for $n\geq 2$).
    \end{itemize}
\end{example}
\begin{definition}
    For $g\in G$, $x = 1, 2, 3, \dots$,
    \[g^x = \underbrace{g\circ g\circ \cdots \circ g}_{x\text{ times}}\]
    We extend this to define $g^0 = e$ and $g^{-n} = (g^n)^{-1}$ (so that our usual exponent rules also apply).
\end{definition}
\begin{example}
    From just now, in $(\ZZ/N\ZZ, +)$, $1^3 = 3$.
\end{example}
\begin{definition}[Element Order]
    The smallest (positive) $n$ with $g^n = e$ is called the \ul{order} of $g$.

    If there is no such $n$, we say $g$ has \ul{infinite} order.
\end{definition}
\begin{proposition}
    If $G$ is a finite group, then every element $g\in G$ has finite order.
\end{proposition}
\begin{proof}
    Consider all powers of $g$
    \[g, g^2, g^3, g^4, \dots\]
    so at some point, we will have
    \[g, g^2, g^3, g^4, \dots, g^i, \dots, g^j, \dots\]
    where $g^i$ and $g^j$ are equal. Then $g^{j-i} = e$. Hence $G$ has a finite order.
\end{proof}
\begin{proposition}
    Let $g\in G$ have order $k$, with $g^n = e$. Then $k\mid n$.
\end{proposition}
\begin{proof}
    We use the division algorithm. We write
    \[n = q\cdot k + r\text{ with }0\leq r < k\]
    then we have
    \[e = g^n = (g^k)^q g^r = e^q\cdot g^r\]
    so $g^r = e$, which forces $r = 0$ since $0\leq r < k$. So $n = qk \Rightarrow k\mid n$.
\end{proof}
\begin{theorem}
    $g^{\#G} = e$. In particular, $\ord g\mid \#G$.
\end{theorem}
\begin{proof}[Proof for Abelian groups]
    Let $G = \{g_1, \dots, g_n\} = \{gg_1, gg_2, \dots, gg_n\}$. No two are equal, since we can take inverse of $g$. We multiply them all together:
    \begin{align*}
        g_1g_2\cdots g_n          & = (gg_1)\cdots (gg_n)        \\
        \cancel{g_1g_2\cdots g_n} & = \cancel{g_1\cdots g_n} g^n \\
        e                         & = g^n
    \end{align*}
    so we have as desired.
\end{proof}
This is true even if $G$ is not Abelian - it's Lagrange's Theorem, which we won't cover here\footnote{Covered in Math 1530, Abstract Algebra.}.

Note that our previous cryptosystems: Diffie-Hellman key exchange and Elgamal, works in \emph{any} group.

\textbf{Q: Why would we want to be able to pick our group?}

Might we want to do this in a group that allows for fast operations? That makes encryption and decryption easy, but it also makes computing the discrete log difficult. We want groups that are \emph{easy enough} and \emph{hard enough}. We might appreciate this by the end of the course\dots

\subsection{Computation Complexity}
How might we quantify ``easy'' or ``hard'' in cryptography.
\begin{example}
    Let $g\in G$ a group. Let's consider exponentiation
    \[x\longmapsto g^x\]
    if $x$ has $k$ bits (i.e. $x\approx 2^k$). How many steps does it take us to compute $g^x$? At most $2k$ multiply and add steps.

    What about solving the discrete log problem:
    \[g^x\longmapsto x\]
    where $x$ has $k$ bits. How many steps does this take (na\"ively, trying every power)? About $2^k$ steps.
\end{example}
\begin{definition}[Big-O]
    We say $f(x) = \mathcal{O}(g(x))$ if there are \emph{constants} $c$ and $c'$ with
    \[f(x) \leq c\cdot g(x) \quad \text{for all }x\geq c'\]
\end{definition}
\begin{example}
    Say $f(x) = \mathcal{O}(1) \Leftrightarrow f$ is bounded.
\end{example}

If $f(x) = \mathcal{O}(x^c)$, then we say this is a ``easy'' problem.\footnote{We take $x$ to be the number of bits of the input}

If $f(x) = \mathcal{O}(c^x)$, we think of this as a ``hard'' problem.

\begin{proposition}
    If
    \[\lim_{x\to \infty}\frac{f(x)}{g(x)} < \infty\]
    then $f(x) = \mathcal{O}(g(x))$.
\end{proposition}
\begin{proof}
    Using definition of limits, for any $\varepsilon > 0$:
    \[\left\vert\frac{f(x)}{g(x)} - L\right\vert < \varepsilon \text{ for }x\geq c\]
    then $f(x) < (L + \varepsilon)\cdot g(x)$.
\end{proof}
\begin{example}
    $2x^2+5x+7 = \mathcal{O}(x^2)$.
\end{example}
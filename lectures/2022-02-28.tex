%!TEX root = ../notes.tex
\section{February 28, 2022}
\subsection{Miller-Rabin Primality Test}
Recall \cref{prop:aq-1-mod-p} from last class.
\begin{proposition*}
    Let $p$ be an odd prime. Write
    \[p-1 = 2^k\cdot q\quad\text{with $q$-odd}\]
    Then either
    \[a^{q}\equiv 1\mod p\]
    or one of $a^q, a^{2q}, a^{4q},\dots, a^{2^{k-1}q}$ is $\equiv -1\mod p$.
\end{proposition*}

For $561$, we write $561 = 2^4\cdot 35$.
\begin{align*}
    2^{35}  & \equiv 263 \pmod{561} \\
    2^{70}  & \equiv 166 \pmod{561} \\
    2^{140} & \equiv 67 \pmod{561}  \\
    2^{280} & \equiv 1 \pmod{561}
\end{align*}
using the following code\dots
\begin{lstlisting}[language=Python]
from crypto import pow_mod

pow_mod(2, 35) # 263
263 ** 2 % 561 # 166
166 ** 2 % 561 # 67
67 ** 2 % 561 # 1
\end{lstlisting}

\begin{definition}[Miller-Rabin Witness]
    $a$ is a \ul{Miller-Rabin witness} if $a$ does not satisfy above proposition.
\end{definition}

\begin{theorem}
    If $n$ is composite, then at at least $75\%$ of $a\in(\ZZ/n\ZZ)$ are Miller-Rabin witnesses.
\end{theorem}
\begin{proof}
    Given on faith.
\end{proof}

\begin{algorithm}[Miller-Rabin Probabilistic Primality Test]
    ~\begin{lstlisting}[numbers=none,language=Python,escapeinside={(*}{*)}]
from random import randrange
from crypto import pow_mod

def miller_rabin(n, a):
    """
    Miller-Rabin primality test on number n and base a
    """
    q, k = n - 1, 0
    # Write (*$n-1 = 2^k\cdot q$*)
    while q % 2 == 0:
        k = k + 1
        q = q // 2
    a = pow_mod(a, q, n)
    if a == 1 or a == n - 1:
        return False
    for _ in range(k - 1):
        a = a ** 2 % n
        if a == n - 1:
            return False
    return True

def is_prime(n):
    for _ in range(50):
        if miller_rabin(n, randrange(1, n)):
            return False
    return True
\end{lstlisting}
\end{algorithm}

Using this, we can find large prime numbers using
\begin{lstlisting}[language=Python]
from crypto import miller_rabin, is_prime
def next_prime(n):
    while not is_prime(n):
        n = n + 1
    return n
\end{lstlisting}

\begin{theorem}[Prime Number Theorem]
    Probability that $n$ is prime is about
    \[\frac{1}{\log(n)}\]
    More formally,
    \[\lim_{x\to \infty}\frac{\text{\# of primes $\leq x$}}{x/\log x} = 1\]
\end{theorem}

So the Miller-Rabin test lets us efficiently find (large) prime numbers.
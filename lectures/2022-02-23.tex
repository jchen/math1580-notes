%!TEX root = ../notes.tex
\section{February 23, 2022}
\subsection{Chinese Remainder Theorem}
\recall (HW2, Q2) asked us to find $x$ with
\begin{align*}
    x & \equiv 3\mod 7 \\
    x & \equiv 4\mod 9
\end{align*}
We solved this by setting
\begin{align*}
    x = 7y + 3 & \equiv 4\mod 9         \\
    7y         & \equiv 1\mod 9
    \intertext{and taking $7^{-1}\pmod{9}$ which is $4$. So we have}
    y          & \equiv 4\mod 9         \\
    x = 7y+3   & \equiv 31\text{ works}
\end{align*}
\begin{theorem}[Chinese Remainder Theorem]
    Let $\{m_i\}$ be a set of pairwise coprime numbers. That is, $\gcd(m_i, m_j) = 1$ for $i = j$. Then the system
    \begin{align*}
        x & \equiv a_1\mod m_1 \\
        x & \equiv a_2\mod m_2 \\
          & \vdots             \\
        x & \equiv a_n\mod m_n
    \end{align*}
    has a solution.
\end{theorem}
\begin{proof}
    By induction on $n$.
    \begin{description}
        \item[Base case.] $n = 1$, then we take $x = a_1$ which is a solution.
        \item[Mini inductive step.] We first solve for $n=2$. We have
            \begin{align*}
                x & \equiv a_1\mod m_1 \\
                x & \equiv a_2\mod m_2
            \end{align*}
            Let $u_1\cdot m_1 + u_2\cdot m_2 = 1$ (by Bezout's identity). Consider quantity
            \begin{align*}
                u_1m_1a_2 + u_2m_2a_1 & \equiv 0 + 1\cdot a_1 \equiv a_1 \mod m_1 \\
                                      & \equiv 1\cdot a_2 + 0 \equiv a_2\mod m_2
            \end{align*}
            which solves for $n=2$.
        \item[Full inductive step.] Let $n\geq 3$, we solve equation
            \begin{align*}
                x & \equiv a\mod m_1m_2\cdots m_{n-1} \\
                x & \equiv a_{n}\mod m_{n}
            \end{align*}
            where the solution $a$ from the first equation comes from our inductive hypothesis. We can solve this by our mini inductive step (the $n=2$ case).
    \end{description}
    Which concludes this proof.
\end{proof}

\begin{example}
    The Chinese Remainder Theorem allows us to solve congruences with composite moduli. For example, solve
    \[x^2\equiv 18 \mod 21\]
    This is equivalent to solving
    \begin{align*}
        x^2 & \equiv 18 = 0\mod 3 \\
        x^2 & \equiv 18 = 4\mod 7
    \end{align*}
    since $21 = 3\cdot 7$. So this is equivalent to solving
    \begin{align*}
        x & \equiv 0\mod 3     \\
        x & \equiv \pm 2\mod 7
    \end{align*}
    Using CRT, we have
    \[1\cdot 7 + (-2)\cdot 3 = 1\]
    so
    \[x = 0\cdot 1\cdot 7 + (-2)\cdot 2\cdot 2 = -12 \equiv 9\mod 21\]
    we do check that $9^2 = 81\equiv 18\mod 21$.
\end{example}

In general, we claim that square roots mod $p$ are easy to compute.
\begin{proposition}
    Let $p\equiv 3\mod 4$ and $a$ is a square in $\ZZ/p\ZZ$ (that is, $x^2\equiv a$ has a solution).

    Then
    \[x\equiv a^{\frac{p+1}{4}}\]
    is a solution.
\end{proposition}
\begin{proof}
    Say $a\equiv b^2\mod p$. Then
    \[(a^{\frac{p+1}{4}})^2 = a^{\frac{p+1}{2}}\equiv (b^2)^{\frac{p+1}{2}}\equiv b^{p+1} \equiv b^{p-1}\cdot b^2\equiv 1\cdot b^2 = b^2 \equiv a\mod p\]
    which concludes the proof.
\end{proof}
So we can compute square roots modulo a composite number $N$ by taking the prime factor decomposition of $N$, and taking the square root mod each factor, then using CRT.

Conversely, any efficient algorithm to find square roots mod $N$ can be used to factor $N$.

\emph{Why?}
\begin{enumerate}[1.]
    \item We generate an element $x$ mod $N$.
    \item Ask for square root of $x^2$.
    \item Good chance that we get $y\not\equiv \pm x$.
          We now have
          \begin{align*}
              x^2        & \equiv y^2\mod N \\
              (x+y)(x-y) & \equiv 0\mod N
          \end{align*}
    \item We can now calculate $\gcd(x+y, N)$ and find some factors of $N$.
\end{enumerate}

\subsection{Euler's Theorem}
\recall Fermat's Little Theorem which says that
\begin{align*}
    a^{p-1} & \equiv 1\mod p\quad \text{ for }p\nmid a
    \intertext{What happens when we replace $p$ with $N$ where $N$ is composite?}
    a^{N-1} & \overset{?}{\equiv}1\mod N\quad \text{ for }\gcd(N,a)=1
\end{align*}
No! Recall demo showing counterexample.

\begin{proposition}
    If $N = pq$ for primes $p$ and $q$, then
    \[a^{(p-1)(q-1)}\equiv 1\mod N\quad\text{ for }\gcd(N, a) = 1\]
\end{proposition}
\begin{proof}
    \textsc{wlog} taking mod $p$, we have
    \[a^{(p-1)(q-1)}\equiv (a^{p-1})^{q-1}\equiv 1^{q-1}\equiv 1\mod p\]
    Similarly mod $q$ by symmetry. Then it is congruent to $1\mod pq$.
\end{proof}

We can generalize this...
\begin{proposition}[Euler's Theorem]
    For any composite $N$, we have
    \[a^{\varphi(N)}\equiv 1\mod N\quad\text{ for }\gcd(N, a) = 1\]
\end{proposition}

\subsection{Exponentiation}
Given an exponent $x$, we can compute $e^x$ fast. Inverting this gives us the \emph{Discrete Log Problem}. Diffie-Hellman Key Exchange and Elgamal rely on this.

What if we think of this as a function of the base? Given a base $x$, we want to take it to exponent $e$ to get $x^e$. Inverting this is the \emph{Extracting Roots} problem. This is the basis of the RSA cryptosystem (see next time!).

\begin{claim*}
    Let $\gcd(e, p-1) = 1$. We can construct $de\equiv 1 \mod p-1$. Then $(x^e)^d\equiv x$.
\end{claim*}

So \emph{extracting roots} is easy mod prime $p$, but hard mod composites. We'll see this next time. 
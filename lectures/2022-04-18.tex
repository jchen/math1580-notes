%!TEX root = ../notes.tex
\section{April 18, 2022}
\subsection{Midterm 2 Review}
\begin{itemize}
    \item Apologies for poor communication.
    \item Feedback is welcome---will send out form.
\end{itemize}
\emph{Average:} 28/40.

``A-level work'' would equate to doing around 3 problems, \\
``B-level work'' would equate to doing around 2 problems, and \\
``C-level work'' would equate to doing around 1 problem.


\begin{problem}
\emph{About 3/4 solved.}

\ul{Idea}: exploit something special about $n$ being a Carmichael number. That is that $a^{N-1}\equiv 1\pmod{N}$. We can do something inspired by the Miller-Rabin test. We then have
\[\left( a^{\frac{N-1}{2}}\equiv 1\pmod{N} \right)\]
If $a^{\frac{N-1}{2}}\equiv 1$, then we decrement the exponent by a factor of two. We then check
\[\left( a^{\frac{N-1}{4}}\equiv 1\pmod{N} \right)\]
and so on. Eventually, we'll find a nontrivial square root of $1$. This produces a factorization, since
\begin{align*}
    x^2        & \equiv 1\pmod{n} \\
    (x-1)(x+1) & \equiv 0\pmod{n}
\end{align*}
so $\gcd(x\pm 1, n)$ likely allows us to recover a factor of $N$. We run through this multiple times with different values of $a$.
\end{problem}

\begin{problem}
\emph{Hardest problem, 1/2 solved.}

\ul{Idea}: Implement some reasonably general-purpose factorization method:
\begin{enumerate}
    \item Lenstra's Elliptic Curve Factorization.
    \item Quadratic Sieve.
    \item Pollard $\rho$ method.
\end{enumerate}

Things that would not work:
\begin{enumerate}
    \item Trial division.
    \item Pollard $p-1$.
\end{enumerate}

The factorization was $15\times 35$ digits, which gives a relatively equal runtime for Elliptic Curve and Quadratic Sieve.
\end{problem}

\begin{problem}
\emph{About 2/3 solved (generally speaking, lost 3 points on it).}

\ul{Idea}: solve DLP. Babystep-Giantstep, as in class, yields 7/10 due to excessive memory usage. 4 people solved this problem in a way that didn't use a shit-ton of RAM, with 3 distinct solutions.

\emph{How to solve DLP without using tons of RAM:}
\begin{enumerate}
    \item Babystep-Giantstep with fewer babysteps ($B$ babysteps). We need $B + \frac{N}{B}$ time to solve this, which is minimized if $B\equiv N^{0.5}$. We take a smaller $B\equiv N^{0.3}$ or something, and our time would be $N^{0.7}$, still within the bounds.
    \item Index calculus. Time/Memory are asymptotically small compared to Babystep-Giantstep.
    \item Pollard $\rho$ method \emph{(discussion of which is in textbook)}. Gives runtime of $N^{0.5}$ with minimal memory usage.
\end{enumerate}

\end{problem}

\begin{problem}
\emph{This was a fairly easy problem.}
\end{problem}

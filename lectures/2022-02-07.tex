%!TEX root = ../notes.tex

\section{February 7, 2022}
\subsection{Orders mod \texorpdfstring{$p$}{p}}
\recall If $a\not\equiv 0\pmod{p}$, then we have $a^{p-1}\equiv 1\pmod{p}$, which was \cref{theorem:flt}, Fermat's Little Theorem.

\begin{definition}[Order of $a$ mod $p$]
    The \ul{order} of $a \pmod{p}$ is the smallest positive $k$ such that
    \begin{equation*}
        a^k\equiv 1\pmod{p}
    \end{equation*}
    This is not to be confused with \cref{defn:order-of-prime-factor} which is the power of $p$ in the prime factorization of $a$. This is the order of $a$ in the multiplicative group $\ZZ/p\ZZ$.
\end{definition}

\begin{proposition}
    let $a\in (\ZZ/p\ZZ)^\times$ be of order $k$. If $a^n\equiv 1\pmod{p}$, then $k\mid n$.

    In particular, $k\mid p-1$ by \cref{theorem:flt}, Fermat's Little Theorem.
\end{proposition}
\begin{proof}
    We write $n = k\cdot q + r$ such that $0\leq r < k$ ($\ZZ$ is a Euclidean domain)
    \begin{align*}
        1 & \equiv a^n\equiv a^{kq+r} \equiv (a^k)^q\cdot a^r\equiv a^r
    \end{align*}
    Since $k$ is the minimal positive number such that $a^k\equiv 1$, then this forces $r = 0$. Then $k\mid n$.
\end{proof}

\begin{theorem}[Primitive Root Theorem]
    Let $p$ be prime. Then there is a $g$ such that
    \[(\ZZ/p\ZZ)^\times = \{1, g, g^2, \dots, g^{p-2}\}.\]
    We call $g$ a \ul{primitive root} or \ul{generator}.
\end{theorem}
\begin{example}
    $p = 5$, $(\ZZ/5\ZZ)^\times = \{1, 2, 3, 4\}$.

    1? No: $\{1, 1^2, 1^3\} = \{1\}$

    2? Yes: $\{1, 2, 2^2, 2^3\} = \{1, 2, 4, 3\}$

    3? Yes: $\{1, 3, 3^2, 3^3\} = \{1, 3, 4, 2\}$

    4? No: $\{1, 4, 4^2, 4^3\} = \{1, 4\}$
\end{example}

\begin{remark}
    In general, the number of primitive roots is $\varphi(p-1)$. (Take the group of exponents and solve for power).
\end{remark}

\subsection{Discrete Logarithm Problem}

We go on to discuss a fundamental property about exponentiation mod $p$. Let's fix some $p$ and primitive root $g$.

Given some $a$, we can compute $g^a$ efficiently
\begin{align*}
    a & \longrightarrow g^a \qquad \text{This is \ul{easy}}             \\
    a & \overset{?}{\longleftarrow} g^a \qquad \text{This is \ul{hard}}
\end{align*}

Note that
\[g^a\equiv g^b \Leftrightarrow g^{a-b}\equiv 1 \Leftrightarrow p-1\mid a-b\]
so $a$ is determined mod $p-1$.

\begin{definition}[Discrete Logarithm]
    The \ul{discrete logarithm} of $g^a$ is $a$.
\end{definition}

This is known as the ``Discrete Logarithm Problem'' (DLP), which is concerned with how we can compute discrete logarithms.

This idea is fundamental to computer security! The real-world analogue is if you go to the bank after hours and deposit a check or cash into the deposit slot. It is relatively easy for one to deposit an item but hard for someone who doesn't work at the bank\footnote{Say, possessing a \emph{key} or \emph{password}.} to access that item.

\subsection{Cryptographic Systems}
\subsubsection{Symmetric Cryptography}
We have $3$ people, \emph{Alice}, \emph{Bob}, and \emph{Eve}.

Bob has a message $m$ which he wants to send to Alice. However, everything he sends to Alice can (and is) intercepted by Eve. He wants to encrypt this message $m$ he sends to Alice.

We say that a message $m\in \mathcal{M}$ in the space of possible messages. We have secret key $k\in \mathcal{K}$ that can encrypt $m$ into ciphertext $c\in \mathcal{C}$ in the space of ciphertexts.
\[\left\{\begin{array}{c}
        \text{Message $m\in \mathcal{M}$} \\
        \text{Secret key $k\in\mathcal{K}$}
    \end{array}\right\} \rightsquigarrow \text{Ciphertext $c\in\mathcal{C}$} \longrightarrow \text{Alice} \rightsquigarrow m\]

If we fix $k$, we have
\begin{align*}
    e_k(m) & = e(k, m) \\
    d_k(c) & = d(k, c)
\end{align*}
be our encryption and decryption functions. We usually take $m$ to be a number, and we can encode letters to numbers (0-255) using ASCII.

In Python, this is implemented using functions like \textsf{ord} (character to encoding) and \textsf{chr} (encoding to character).

We'll just talk about transmitting numbers since we can convert freely between them and text.

\textbf{Q: What do we want out of our cryptosystem?}
\begin{enumerate}
    \setcounter{enumi}{-1}
    \item The system is secure even if Eve knows the design. (Assume Eve knows the encryption and decryption functions, but so long as she doesn't know the key).
    \item $e$, the encryption function, is easy to compute.
    \item $d$, the decryption function, is similarly easy to compute.
    \item Given $c_1, c_2, \dots$ a collection of ciphertexts, encrypted with the \emph{same} key $k$, it's hard to compute any message $m_i$.
    \item Given $(m_1, c_1), \dots (m_n, c_n)$ some collection of messages and their encryptions, it remains difficult to compute $d_k(c)$ for $c\not\in \{c_1, \dots, c_n\}$. This is called a \textit{``\ul{chosen plaintext attack}''}. 
\end{enumerate}
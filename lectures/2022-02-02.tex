%!TEX root = ../notes.tex
\section{February 2, 2022}
\subsection{Inverses mod \texorpdfstring{$m$}{m}}
\recall Last time, we showed in \cref{existence-of-inverse} that there exists an integer $b$ with with $a\cdot b\equiv 1\mod m$ iff $\gcd(a, m) = 1$.

\begin{claim}
    We further claim that if such a $b$ exists, then it is unique mod $m$.

    That is, if we have
    \begin{align*}
        a\cdot b_1\equiv 1\pmod{m} \\
        a\cdot b_2\equiv 1\pmod{m}
    \end{align*}
    then we have that $b_1\equiv b_2\pmod{m}$.
\end{claim}
\begin{proof}
    We consider $b_1 ab_2$. We have
    \[b_2\equiv (b_1 a)b_2 = b_2(a b_2) \equiv b_2\]
    all taking mod $m$.
\end{proof}

How, then, could we compute this inverse $b$ efficiently?

Recall that last class, we used the extended Euclidean algorithm to compute the linear combination of $a$ and $m$ efficiently,
\begin{align*}
    1 & = a\cdot u + m\cdot v          \\
      & \equiv a\cdot \boxed{u}\mod{m}
\end{align*}
where $u$ is $b$.

\subsection{Modular Arithmetic \emph{continued}}
\begin{definition}[Ring of Integers mod $m$]
    $\ZZ/m\ZZ = \{0, 1, 2, \dots, m-1\}$ with operations $+, -, \times \pmod{m}$.
\end{definition}
\begin{example}
    $\ZZ/4\ZZ = \{0, 1, 2, 3\}$. We have the following operation tables for $\ZZ/4\ZZ$:
    \[\begin{array}[]{c|cccc}
            + & 0 & 1 & 2 & 3 \\ \hline
            0 & 0 & 1 & 2 & 3 \\
            1 & 1 & 2 & 3 & 0 \\
            2 & 2 & 3 & 0 & 1 \\
            3 & 3 & 0 & 1 & 2 \\
        \end{array}\qquad
        \begin{array}[]{c|cccc}
            \times & 0 & 1 & 2 & 3 \\ \hline
            0      & 0 & 0 & 0 & 0 \\
            1      & 0 & 1 & 2 & 3 \\
            2      & 0 & 2 & 0 & 2 \\
            3      & 0 & 3 & 2 & 1 \\
        \end{array}\]
\end{example}

\begin{definition}[Group of Units mod $m$]
    We have the set of units in $\ZZ/m\ZZ$ as
    \begin{align*}
        (\ZZ/m\ZZ)^\times & = \{a\in \ZZ/m\ZZ\mid \exists b \text{s.t. }a\cdot b \equiv 1\} \\
                          & = \{a\in \ZZ/m\ZZ\mid \gcd(a, m) = 1\}
    \end{align*}
\end{definition}
\begin{example}
    $(\ZZ/4\ZZ)^\times = \{1, 3\}$.
\end{example}

\begin{definition}[Euler Totient Function]
    We have
    \begin{equation*}
        \varphi(m) = \# (\ZZ/m\ZZ)^\times
    \end{equation*}
    which counts the number of units modulo $m$.
\end{definition}
\begin{example}
    $\varphi(4) = 2$.
\end{example}

Let's investigate the properties of units. Let's say $a_1, a_2$ are units. Which of the following have to be units?

\begin{table}[h]
    \centering
    \renewcommand\arraystretch{1.7}
    \begin{tabular}{c|l}
                       & Does this have to be a unit?                                                \\ \hline
        $a_1\cdot a_2$ & \begin{minipage}[t]{0.7\textwidth}
                             \ul{Yes}! \vspace{0.5em}

                             Since $\gcd(a_1, m) = 1$ and $\gcd(a_2, m) = 2$
                             so we have $\gcd(a_1a_2, m) = 1$. We also have $a_1b_1\equiv 1\mod m$
                             and $a_2b_2\equiv 1\mod m$,
                             we have $(a_1a_2)(b_2b_1)\equiv 1\mod m$.
                             \vspace{0.5em}
                         \end{minipage} \\ \hline
        $a_1 + a_2$    & \ul{No}. We have counterexample $m = 4$: $1 + 1$ is not a unit.             \\ \hline
        $a_1 - a_2$    & \ul{Also no}. For any $a$, $a - a = 0$ which is never a unit.
    \end{tabular}
    \renewcommand\arraystretch{1.05}
\end{table}

\begin{definition}[Prime Number]
    An integer $n\geq 2$ is \ul{prime} if its only (positive) divisors are $1$ and $n$.
\end{definition}
\begin{example}
    Numbers like $2, 3, 5, 7, 11, 12, \dots$.
\end{example}

What if $m$ is a prime number? Then we have
\[(\ZZ/m\ZZ)^\times = \{1, 2, \dots, m-1\}\]
so we can divide by elements of $\ZZ/m\ZZ$, just like in $\QQ, \RR, \CC$. We can divide by any nonzero element of $\ZZ/m\ZZ$. We call these \ul{fields}!

\subsection{Fast\emph{ish} Powering}
\begin{problem*}
    How might we compute $g^a\mod m$?
\end{problem*}
A na\"ive solution might be
\begin{lstlisting}[language=Python]
def pow_mod(g, a, m): 
    return g ** a % m
\end{lstlisting}
What if we tried to compute \textsf{pow\_mod(239418762304, 12349876234, 12394876123482783641)} or something of the like? Something like this\dots

\begin{center}
    \includegraphics[height=5cm]{images/broken_laptop.png}
\end{center}

We could do something a bit more clever, like taking a mod every time we multiply:
\begin{lstlisting}[language=Python]
def pow_mod(g, a, m): 
    p = 1
    for i in range(a): 
        p = (p * q) % m
    return p
\end{lstlisting}

Yet we \emph{still} couldn't do \textsf{pow\_mod(239418762304, 12349876234, 12394876123482783641)} since that takes the amount of time proportional to \textsf{a}\footnote{Which can become big\dots}.

\begin{example}
    Let's try to compute $3^{37}$ by hand.
    \begin{align*}
         & 3^1                                   & \equiv & \ 3  \mod{100} \\
         & 3^2                                   & \equiv & \ 9  \mod{100} \\
         & 3^4 = (3^2)^2 =                       & \equiv & \ 81 \mod{100} \\
         & 3^8 = (3^4)^2 = 81^2 = 6561           & \equiv & \ 61 \mod{100} \\
         & 3^{16} = (3^8)^2 \equiv 61^2 = 3721   & \equiv & \ 21 \mod{100} \\
         & 3^{32} = (3^{16})^2 \equiv 21^2 = 441 & \equiv & \ 41 \mod{100}
    \end{align*}
    Since $37 = 32 + 4 + 1$, we can simply do
    \[3^{37} = 3^{32} \cdot 3^{4} \cdot 3^{1} = 41\cdot 81\cdot 3=1863\equiv 63 \mod{100}\]
\end{example}
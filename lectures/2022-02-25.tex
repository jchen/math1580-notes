%!TEX root = ../notes.tex
\section{February 25, 2022}
\recall Midterm discrete log problem where
\[x\to x^e\mod p\]
and a solution we gave was take $(x^e)^p\equiv x$ where $de\equiv 1\mod (p-1)$.

What if we didn't take this mod $p$, but instead took it mod $pq$.

We can take the analog of Fermat's Little Theorem mod $pq$, where
\[a^{(p-1)(q-1)}\equiv 1\mod pq\]

What if we did the same thing, instead of taking inverse mod $p-1$, we took it mod $(p-1)(q-1)$ to extract $e^\mathrm{th}$ roots if we know $(p-1)(q-1)$.

We know $p$ and $q$, we can easily figure out $(p-1)(q-1)$. Also, if we know $(p-1)(q-1)$, we also know
\[pq-p-q+1\]
We know that $pq = x$ and $p+q = y$, then $pq$ are roots of $t^2-yt+x=0$.

\subsection{RSA Public-Key Cryptography}
Alice generates
\begin{align*}
    p, q                  & \to \text{Two large prime numbers} \\
    e                     & \to \text{``Public exponent''}     \\
    N = pq                & \to \text{``Public modulo''}       \\
    \boxed{(e, N)}        & \to \text{Public key}              \\
    (d, N)                & \to \text{Private key}             \\
    \text{where }d\cdot e & \equiv 1\mod(p-1)(q-1)
\end{align*}

Bob has some message $m$ he wishes to send to Alice. Bob sends $m^e\mod N$ and sends it to Alice.

After receiving this message, Alice can recover $(m^e)^d\equiv m\mod N$.

$p, q$ are private, but $pq$ is private. The security of RSA rests on multiplication being easy, but factorization being hard ($pq$ is hard to factorize into $p$ and $q$).

We might implement such an algorithm like so:
\begin{lstlisting}[language=Python]
from crypto import gcd, ext_gcd, pow_mod
N = p * q
d = ext_gcd(e, (p-1) * (q-1))[0] % ((p-1) * (q-1))
m = 1234567891234786951234010239847123748
# 0 < m < N is True
c = pow_mod(m, e, N)
pow_mod(c, d, N) # => m
\end{lstlisting}

If we were Alice and Bob, we \emph{still} have one step to go! We need to find ourselves big prime numbers $p, q$ (finding $e$ is easy, we can just use our \textsf{gcd} algorithm). How do we do that?

\subsection{Primality Testing}
\textsf{Prime hunting!}

Prime numbers are reasonably common. So generating a large prime number amounts to generating a number, checking if it's prime, and repeating until we get a prime.

This reduces to the problem of checking if a number is prime. Given an $n$, is it a prime?

\begin{lstlisting}
from crypto import pow_mod
n = 123874610239487102893741890237023
pow_mod(2, n-1, n)
\end{lstlisting}
gives us a basic primality check. Fermat's Little Theorem says that if $n$ is prime, then $2^{p-1}\equiv 1\mod n$.

\begin{definition}[Witnesses]
    We say that $a$ is a \ul{witness} for the compositeness of $n$ if
    \[a^{n-1}\not\equiv 1\mod n\]
    and $gcd(a, n) = 1$.
\end{definition}

We have a problem! Fermat's Little Theorem is not an if-and-only-if. We can have numbers that pass this test for almost every base. Take $n = 3\cdot 11\cdot 17 = 561$.

\begin{lstlisting}[language=Python]
from crypto import pow_mod, gcd
for a in range(n):
    if gcd(a, n) == 1:
        print(pow_mod(a, n-1, n))
\end{lstlisting}
gives $1$ for \emph{everything}\footnote{oh no!}

\begin{definition}[Carmichael Number]
    A \ul{Carmichael number} is a composite number with \emph{no} witness of compositeness.
\end{definition}
\begin{example}
    $561$ is a Carmichael Number.
\end{example}

We now \emph{almost} have a way of checking for primality, but it doesn't always quite work.

Since taking to the power of $n-1$ is a group homomorphism, then Lagrange's Theorem states that if there is a witness, then there are \emph{a lot} of witnesses.

\begin{proposition}\label{prop:aq-1-mod-p}
    Let $p$ be an odd prime. Write
    \[p-1 = 2^k\cdot q\quad\text{with $q$-odd}\]
    Then either
    \[a^{q}\equiv 1\mod p\]
    or one of $a^q, a^{2q}, a^{4q},\dots, a^{2^{k-1}q}$ is $\equiv -1\mod p$.
\end{proposition}
\begin{proof}
    We look at this sequence
    \[a^q, a^{2q}, a^{4q},\dots, \underbrace{a^{2^iq}}_{\not\equiv 1\pmod p}, \underbrace{a^{2^{i+1}q}}_{\equiv 1\pmod p}, \dots, a^{2^{k-1}q}, a^{2^kq}\]
    We have that $a^{2^kq}\equiv 1\mod p$ by Fermat's Little Theorem. There's some point where we `become' congruent to $1\mod p$. If the first one is one, then we have the first case. Otherwise we have $a^q\not\equiv 1$ mod $p$, then we repeatedly square until we get to $1\mod p$. Then \emph{right before} we turned to $1\mod p$, we would have had $-1\mod p$. In our example above, this is $a^{2^iq}$. 
\end{proof}
%!TEX root = ../notes.tex
\section{March 11, 2022}
\subsection{Quadratic Sieve \emph{continued}}
We resume our discussion of the example of \cref{example:quadratic-sieve}. We want to find an optimal $B$ for the algorithm, and we do this by analyzing runtimes.

The runtime of $X$ is first approximately $B\cdot u^u$ where \[u = \frac{\log \sqrt{N}}{\log B}.\]

And in addition, solving a $B\cdot B$ system of linear equations. It takes $B^2$ operations to zero out one column ($B$ rows and subtracting a column takes $B$ operations), then we do this for $B$ columns. So the runtime is $B^3$.

We now have $u^u$ is decreasing in $B$ and $B^3$ is increasing in $B$. Minimizing the runtime, we set $B^3 \sim B\cdot u^u$, i.e. $B^2\sim u^u$. Using very sketchy mathematics,
\begin{align*}
    u^u                        & \sim B                             \\
    u\log u                    & \sim B                             \\
    \frac{log N}{\log B}\sim u & \sim B                             \\
    \log B                     & \sim \sqrt{\log N}                 \\
    B                          & \sim e^{\sqrt{\log N}}             \\
    u                          & \sim \frac{\log{\sqrt{N}}}{\log B}
    \sim \frac{\log N}{\log B} \sim \sqrt{\log N}
\end{align*}
is a \emph{loose} guess. But we can use this to make a more approximate guess.

Being more rigorous, $u^u= B^2$ gives $u\log u = 2\log B$. Then using our estimate for $u$ from above, we have
\begin{align*}
    u\cdot \log\left[\sqrt{\log N}\right] & = 2\log B                                    \\
    \frac{1}{2}\frac{\frac{1}{2}\log N}{\log B}\log\log N
    = \frac{1}{2}u\log\log N              & = 2\log B                                    \\
    (\log B)^2                            & = \frac{1}{8}\log N\log\log N                \\
    \Rightarrow B                         & \sim e^{\sqrt{\frac{1}{8}\log N \log\log N}}
\end{align*}
where we note the difference of a factor of $\frac{1}{8}\log\log N$ isn't that far off from $e^{\sqrt{\log N}}$. So total runtime is around $B^3$ which is $e^{\sqrt{\frac{9}{8}\log N\log\log N}}$. \emph{It's not super fast but not totally stupid.}

Realistically, the $B^3$ can be reduced to $B^2$ in solving our $B\times B$ system, but we can get rid of our factors of $2$'s from before. This lets us get rid of the factor of $\frac{9}{8}$ so our total runtime is like
\[e^{\sqrt{\log N\log\log N}}\]
There are even faster algorithms, namely the number field sieve. It replaces the square root from above into a cube root and has runtime
\[e^{\sqrt[3]{c\cdot \log N(\log\log N)^2}}\]

\subsection{Index Calculus \& Discrete Logs}
The discrete log problem is solving $x$ given $g^x = a$ and we know $g$ and $a$. We can do a similar strategy to the quadratic sieve.
We calculate
\begin{align*}
    g^1 \pmod p & = 2                   \\
    g^2 \pmod p & = \text{(not smooth)} \\
    g^3 \pmod p & = 2\cdot 3            \\
\end{align*}
where we pick some smoothness bound $B$. We then find $g^i\equiv \text{(B-smooth)}\mod p$. We find enough $B$ smooth things such that by linear algebra, we can solve $g^x = 2, 3, 5, 7, \dots$.

We take
\begin{align*}
    a\mod p             & = \text{(not smooth)} \\
    a\cdot g^{-1}\mod p & = \text{(not smooth)} \\
    a\cdot g^{-2}\mod p & = 3
\end{align*}
which is smooth. So we find $a\cdot g^{-i}\equiv \text{(B-smooth)}\mod p$. Then by linear algebra we can solve for $a$ given lots of smooth $g^i$.

\begin{ques*}What is the runtime of this? \end{ques*}

We need to go up to $g^x$ where $x$ is about $B\cdot u^u$ ($u^u$ is the probability of being $B$-smooth). Linear algebra will take runtime $B^2$. Similar to just now, this is exactly the same problem as above except $\sqrt{N}$ is replaced with $P$. Before (in the quadratic sieve), we had
\[B\sim \exp\left(\frac{1}{4}\log N\log\log N\right)\]
and
\[\text{Runtime}\sim \exp\left(\log N\log\log N\right)\]
Replacing $\log{N}$ with $2\log P$, our bound becomes
\[B\sim \exp\left(\frac{1}{2}\log p\log\log p\right)\]
\[\text{Runtime}\sim \exp\left(2\log p\log\log p\right)\]
What if $g\in$ subgroup of order $q$? Babystep-Giantstep changes from $\sqrt{p}$ to $\sqrt{q}$. With index calculus, there is no advantage of knowing that $g$ is im a smaller subgroup. We don't have a decrease in runtime. 
%!TEX root = ../notes.tex
\section{January 31, 2022}
\subsection{Linear Combinations \emph{continued}}
Recall from last time that we proposed that
\[\text{greatest common divisor} \leq \text{least linear combination}.\]
\begin{example}
    $\gcd(2024, 748)=44$ because we have
    \begin{align*}
        2024 & = 748\cdot 2 + 528                                  \\
        748  & = 528\cdot 1 + 220                                  \\
        528  & = 220\cdot 2 + 88                                   \\
        220  & = 88\cdot 2 + \boxed{44} \leftarrow \gcd(2024, 748) \\
        88   & = 44\cdot 2 + 0
    \end{align*}
    We determine which linear combinations or $2024$ znd $748$ we can create:
    \begin{align*}
        2024 & = 1\cdot 2024 + 0\cdot 748                         \\
        748  & = 0\cdot 2024 + 1\cdot 748                         \\
        528  & = 1\cdot 2024 + (-2)\cdot 748                      \\
        220  & = 748 - 1\cdot 528                                 \\
             & = 748 - 1\cdot (1\cdot 2024 + (-2)\cdot 748)       \\
             & = -1\cdot 2024 + 3\cdot 748                        \\
        88   & = 528 - 2\cdot 220                                 \\
             & = \underbrace{[1\cdot 2024 + (-2)\cdot 748]}_{528}
        - 2\cdot \underbrace{[-1\cdot 2024 + 3\cdot 748]}_{220}   \\
             & = 3\cdot 2024 - 8\cdot 748                         \\
        44   & = 220 - 2\cdot 88                                  \\
             & = [-1\cdot 2024 + 3\cdot 748]
        - 2\cdot [3\cdot 2024 - 8\cdot 748]                       \\
             & = -7\cdot 2024 + 19\cdot 748
    \end{align*}
\end{example}
Following this example, we have shown that every common divisor of $a$ and $b$ can be written as a linear combination of $a$ and $b$, and since the greatest common divisor has to be less than the least linear combination (as shown last time), the greatest common divisor \emph{is} the least linear combination\footnote{Assume for contradiction that the gcd were any less, then that would also be a linear combination. $\lightning$}.

We realize that there is a \emph{recurrence} happening here. If we call every set of coefficients $x, y$ and $z, w$ for $a$ and $b$ respectively, such that
\begin{align*}
    a & = x\cdot a_0 + y\cdot b_0 \\
    b & = z\cdot a_0 + y\cdot b_0
\end{align*}
where $a_0$ and $b_0$ are the original numbers, we can use a sliding window approach\footnote{Updating our iterators on every loop by sliding our window of coefficients down.} again to determine the next set of $x, y, z, w, a, b$.

Recall from last time we had
\begin{align*}
    a' & = b       \\
    b' & = a\mod b
\end{align*}
We can extend this algorithm for our new coefficients:
\begin{align*}
    x' & = z                                               \\
    y' & = w                                               \\
    z' & = w - \left\lfloor\frac{a}{b}\right\rfloor\cdot z \\
    w' & = y - \left\lfloor\frac{a}{b}\right\rfloor\cdot w \\
\end{align*}
where $\left\lfloor\frac{a}{b}\right\rfloor$ are the quotients from our Euclidean Algorithm. Note that initially, we have
\begin{align*}
    a & = 1\cdot a_0 + 0\cdot b_0 \\
    b & = 0\cdot a_0 + 1\cdot b_0
\end{align*}
so we have initial values of $x=1, y=0, z=0, w=0$.

so our code for the \emph{extended Euclidean's Algorithm} is now
\begin{lstlisting}[language=Python]
def ext_gcd(a, b):
    x, y, z, w = 1, 0, 0, 1
    while b != 0:
        x, y, z, w = z, w, w - (a // b) * z, y - (a // b) * w
        a, b = b, a % b
    return (x, y)
\end{lstlisting}

\subsection{Modular Arithmetic}
% *** Create recall environment
\recall We used a substitution/shift cipher to encrypt text:
\[\begin{array}{ccc}
        \mathtt{Y} & \mathtt{E} & \mathtt{S} \\
        \downarrow & \downarrow & \downarrow \\
        \mathtt{D} & \mathtt{J} & \mathtt{X}
    \end{array}\]
by incrementing 5 letters for each lecture.

$a = 0$, $b = 1$, \dots, $z=25$.

We had this notion of
\begin{align*}
    \text{ciphertext} & = \text{plaintext} + 5 \\
    \mathtt{d}        & = \mathtt{y} + 5       \\
    3                 & = 24+5 = 29
\end{align*}

\begin{definition}
    We say $a\equiv b\mod{m}$ if $m\mid a - b$.

    We say ``$a$ is congruent\footnote{Congruence is a ``behave like'' equality.} to $b$ modulo $m$''.
\end{definition}
\begin{example}
    \begin{align*}
        24 + 5 \equiv 3\mod{26} \\
        22 + 2 \equiv 1\mod{12} \\
    \end{align*}
    The first example is from our shift sipher, the second example is equivalent to ``two hours after 11:00, it is 1:00''.
\end{example}

\begin{proposition}
    If we have
    \begin{align}
        a_1          & \equiv a_2\mod{m} \nonumber                    \\
        b_1          & \equiv b_2\mod{m} \nonumber                    \\
        \intertext{Then we have the following: }
        a_1 + b_1    & \equiv a_2 + b_2\mod{m} \label{premod-add}     \\
        a_1 - b_1    & \equiv a_2 - b_2\mod{m} \label{premod-sub}     \\
        a_1\cdot b_1 & \equiv a_2\cdot b_2\mod{m} \label{premod-mult}
    \end{align}
\end{proposition}
\begin{proof}
    For \cref{premod-add}, realize that we have
    \[(a_1 + b_1) - (a_2 + b_2) = (a_1 - a_2) + (b_1 - b_2)\]
    and the two terms on the right are each divisible by $m$ by our premise. We can also write out
    \begin{align*}
        a_1 + b_1 & = (a_2 + \alpha m) + (b_2 + \beta m)     \\
                  & = (a_2 + b_2) + (\alpha + \beta)\cdot m.
    \end{align*}
    Similarly, for \cref{premod-sub}, we have
    \begin{align*}
        a_1 - b_1 & = a_2 + \alpha m - (b_2 + \beta m)      \\
                  & = a_2 - b_2 + (\alpha - \beta) \cdot m.
    \end{align*}
    and for \cref{premod-mult}, we have
    \begin{align*}
        a_1\cdot b_1 & = (a_2 + \alpha m)\cdot (b_2 + \beta m)                           \\
                     & = a_2\cdot b_2 + \alpha m b_2 + \beta m a_2 + \alpha\beta m^2     \\
                     & = a_2\cdot b_2 + (\alpha b_2 + \beta a_2 + \alpha\beta m)\cdot m.
    \end{align*}
    which concludes the proofs of the premod rules.
\end{proof}

\begin{proposition}
    There exists $b$ with
    \[a\cdot b\equiv 1\mod{m}\]
    if and only if $\gcd(a, m) = 1$.
\end{proposition}
\begin{proof}
    We can write linear combination equation
    \[a\cdot b + m\cdot k = 1\]
    and we have that the following are equivalent (we cascade down the list and can easily prove the iff relations):
    \begin{enumerate}[i.]
        \item such a $b$ exists,
        \item there is a solution $b, k$ to this equation,
        \item $1$ is a linear combination of $a$ and $m$,
        \item $1$ is the \emph{least} linear combination of $a$ and $m$,
        \item $1 = \gcd(a, m)$.
    \end{enumerate}
    so we have that $1 = \gcd(a, m)$ if and only if $a$'s inverse $b$ exists.
\end{proof}
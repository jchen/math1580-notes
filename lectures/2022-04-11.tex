%!TEX root = ../notes.tex
\section{April 11, 2022}
\subsection{Merkle-Hellman Public Key Cryptosystem}
Alice picks a superincreasing sequence $\boxed{r_1, r_2, \dots, r_n}$. Alice also picks $\boxed{A, B}$ relatively prime integers with $B > 2\cdot r_n$. This is Alice's private key.

Alice computes sequence
\[\boxed{M_i = A\cdot r_i\mod B}\]
which is her public key.

Bob encrypts message $(x_1, \dots, x_n)$ by calculating
\[c = \sum x_i M_i\]
and sending it to Alice.

Alice decrypts $c$ as follows.
\begin{align*}
    c             & = \sum x_i M_i = \sum x_i Ar_i \mod B \\
    A^{-1}\cdot c & = \sum x_i r_i \mod B
\end{align*}
where $x_i\cdot r_i$ is less than $B$. So decryption is to calculate $c' = A^{-1}\cdot c \pmod{B}$. Then write
\[c' = \sum x_i\cdot r_i\]
using algorithm for superincreasing sequences.

We implement as follows:
\lstinputlisting[]{code/merkle-hellman.py}

\subsection{Merkle-Hellman \& Lattices}
We consider the lattice $L$ generated by
\begin{align*}
    V_1     & = (2, 0, 0, \dots, 0, M_1) \\
    V_2     & = (0, 2, 0, \dots, 0, M_2) \\
            & \vdots                     \\
    V_n     & = (0, 0, 0, \dots, 2, M_n) \\
    V_{n+1} & = (1, 1, 1, \dots, 1, c)
\end{align*}
where $c$ is our ciphertext. We note that $L$ contains
\[\left( \sum_{i=1}^n x_iv_i \right)- v_{n+1}\]
\[(2x_1 - 1, 2x_2 - 1, \dots, 2x_n - 1, 0) = (\pm 1, \pm 1, \dots, \pm 1, 0)\]
This vector is \emph{short}! It has length about $\sqrt{n}$. The upshot is that we can reduce breaking M-H to finding short vectors in lattices.

\subsection{Vector Spaces and Inner Products: \emph{Review}}
Let $V\subseteq \RR^m$ be a subspace of dimension $n$. We must have $n\leq m$. This subspace has a basis
\[\left( v_1, v_2, \dots, v_n \right).\]
Any vector $w\in V$ can be written as $w = a_1v_1 + \cdots a_nv_n$ uniquely. Let's say we have
\begin{align*}
    w_1 & = a_{11}v_1 + \cdots a_{1n}v_n \\
    w_2 & = a_{21}v_1 + \cdots a_{2n}v_n \\
        & \vdots                         \\
    w_n & = a_{n1}v_1 + \cdots a_{nn}v_n
\end{align*}
When is $\{w_i\}$ a basis? It is when we can also express
\begin{align*}
    v_1 & = a_{11}w_1 + \cdots a_{1n}w_n \\
    v_2 & = a_{21}w_1 + \cdots a_{2n}w_n \\
        & \vdots                         \\
    v_n & = a_{n1}w_1 + \cdots a_{nn}w_n
\end{align*}
That is, we can define a change-of-basis matrices
\begin{align*}
    A & = \begin{pmatrix}
              a_{11} &        &        \\
                     & \ddots &        \\
                     &        & a_{nn}
          \end{pmatrix} \\
    B & = \begin{pmatrix}
              b_{11} &        &        \\
                     & \ddots &        \\
                     &        & b_{nn}
          \end{pmatrix}
\end{align*}
where $A\cdot B = I$. We also have that $(\det A)(\det B) = 1$ so $\det A \neq 0$. Conversely, Cramer's rule gives a formula for $A^{-1} = \frac{1}{\det A}(\cdots)$ that involves division by $\det A$.

So we have that $\{w_i\}$ is a basis $\iff$ $\det A \neq 0$.

\begin{ques*}
    What if instead of talking about subspaces of a vector space, we talked about a lattice?
\end{ques*}

We restrict $a_i\in\ZZ$. When is $\{w_i\}$ a basis for the same lattice? The logic is the same as saying $A\cdot B = I$ for some integer matrix $B$. This certainly implies $\det A = \det B = 1$. Then $\det A = \pm 1$.

For lattices: $\{w_i\}$ is a basis if and only if $\det A = \pm 1$. 
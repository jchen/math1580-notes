%!TEX root = ../notes.tex
\section{April 6, 2022}
\subsection{Lattices and Cryptography}
The advantage of lattice based cryptography is that they are quantum resistant.

We'll use a \emph{toy example} as a warm-up\footnote{This isn't even secure classically.}.
\begin{example}
    Suppose we have some $q$ that is public knowledge (any integer).

    Alice will choose reasonably small numbers $f, g$
    \begin{align*}
        0                  & < f < \sqrt{\frac{q}{2}} \\
        \sqrt{\frac{q}{4}} & < g < \sqrt{\frac{q}{2}}
    \end{align*}
    which will constitute her private keys. She'll compute $h = f^{-1}\cdot g\pmod{q}$ which will be her public key. We'll assume that $f, g, q$ are all pairwise relatively prime.

    Bob, encrypting message $m$, satisfying $0 < m < \sqrt{\frac{q}{4}}$:
    \begin{enumerate}
        \item Choose random $r$ with $0 < r < \sqrt{\frac{q}{2}}$.
        \item Compute ciphertext $c = r\cdot h + m\pmod{q}$ to send to Alice.
    \end{enumerate}

    Alice, to decrypt the message, will do the following:
    \begin{enumerate}
        \item Calculate $a \equiv f\cdot c\pmod{q}$.
        \item Calculate $b \equiv f^{-1}a\pmod{g}$.
    \end{enumerate}
    Why does this work?
    \begin{align*}
        a \equiv f\cdot c \equiv f(r\cdot h + m) & \equiv f(r\cdot f^{-1}\cdot g + m) \\
                                                 & \equiv r\cdot g + f\cdot m
    \end{align*}
    and we rely on the fact that $r\cdot g + f\cdot m < q$ since
    \[0 < rg + fm < \sqrt{\frac{q}{2}}\sqrt{\frac{q}{2}} + \sqrt{\frac{q}{2}}\sqrt{\frac{q}{4}} \leq q\]
    Thus $a\equiv rg + fm$ (exactly!). Then $b\equiv f^{-1}a\equiv f^{-1}(rg + fm)\pmod{g}\equiv m\pmod{g}$. $m < \sqrt{\frac{q}{4}} < g$ thus $b = m$ exactly.
\end{example}

We can implement this in code (again):
\begin{lstlisting}
from gcd import *

q = 320984712309487123509238471251

while True:
    f = randrange(int(sqrt(q/2)))
    g = randrange(int(sqrt(q/4))+ 1, int(sqrt(q/2)))
    if gcd(f, g) == 1 and gcd(f, q) == 1:
        break

h = (ext_gcd(f, q)[0] * g) % q

def e(m):
    r = randrange(sqrt(q/2))
    c = (r * h + m) % q
    return c

def d(c):
    a = (f * c) % q
    b = (ext_gcd(f, g)[0] * a) % g
    return b
\end{lstlisting}

\begin{ques*}
    What does Eve need to do?
\end{ques*}
Eve knows $q, h$ and wants to figure out $f, g$ with $f\cdot h\equiv g\pmod{q}$ and $f, g$ small ($\mathcal{O}(\sqrt{q})$).

We write this as a vector equation:
\begin{align*}
    f\cdot
    \begin{pmatrix}
        1 \\ h
    \end{pmatrix} - r\cdot
    \underbrace{\begin{pmatrix}
                        0 \\ q
                    \end{pmatrix}}_{V_1} =
    \underbrace{\begin{pmatrix}
                        f \\ g
                    \end{pmatrix}}_{V_2}
\end{align*}
where $f, r$ are unknown integers and $\begin{pmatrix}
        f \\ g
    \end{pmatrix}$ is an unknown vector. The known vectors are on the left.

Our goal is to find a \ul{short vector} in
\[\left\{ a_1v_a + a_2v_2\mid a_1, a_2\in\ZZ \right\}\]

\begin{definition}[Lattice]
    Let $v_1, v_2, \dots, v_n \in \RR^m$ be linearly independent vectors vectors (so $n\leq m$).

    The \ul{lattice} generated by $v_1, \dots, v_n$ is:
    \[\left\{ a_1v_1 + \cdots + a_nv_n : a_1, \dots, a_n\in\ZZ \right\}\]
\end{definition}

\begin{remark}
    There is a fast algorithm to find \emph{short vectors} in a 2D lattice (which is the one above). Then, Eve can use this to break the above cryptosystem. \emph{(We will see this later.)}
\end{remark}

\subsection{Subset Sum Cryptosystem}
The subset sum problem is as follows: Given $M_1, M_2, \dots, M_n\in\ZZ$: find a subset whose sum is $S$.

\begin{example}
    We take $M = \{2, 3, 5, 8\}$ and $S = 10$. $S = 2 + 8 = 2 + 3 + 5$. (We note that the subset is not necessarily unique).
\end{example}

The idea is as follows:

Alice chooses $M_1, M_2, \dots, M_n$. Bob has a message $x_1, x_2, \dots, x_n$ where each $x_i \in \{0, 1\}$. Bob computes $\sum x_i M_i$ (that is, Bob's message specifies a subset of $M_i$'s) and sends to Alice.

Alice has to recover which subset Bob sent her. Alice needs to have chosen $M_1, \dots, M_n$ so that a) the solution to subset-sum is unique, b) it has some secret structure to solve subset sum\dots

\emph{Continued next time. }
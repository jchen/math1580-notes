%!TEX root = ../notes.tex
\section{March 7, 2022}
\subsection{Pollard's \texorpdfstring{$p-1$}{p-1} Method}
\recall the Pollard's $p-1$ method from last time. Let
\[N = p\cdot q\]
\[\begin{array}{c}
        p-1\mid n! \\
        q-1\nmid n!
    \end{array}\rightarrow \gcd(N, a^{n!}-1) = p\]

\begin{lstlisting}[language=Python]
from pollard import * 

def factor(N):
    # Naive, brute force factoring algorithm.
    i = 2
    while N % i != 0:
        i += 1
    return i

def factor(N, a=2):
    # Pollard's method
    i = 1
    while gcd(a - 1, N) == 1:
        i += 1
        a = pow_mod(a, i, N)
    return gcd(a - 1, N)
\end{lstlisting}

This algorithm is \emph{significantly} faster than random guesses. By how much? We can formalize this intuition of size of prime factors:
\begin{definition}
    We say $n$ is \ul{$B$-smooth} if all prime factors are $\leq B$. We define
    \[\Psi(X, B) = \text{\# of $B$-smooth $n\leq X$}.\]
\end{definition}
\begin{theorem}
    Let $X, B$ increase together. Suppose \[(\log X)^\varepsilon \leq \log B \leq (\log X)^{1-\varepsilon}\]
    Then
    \[\frac{\Psi(X, B)}{X} = u^{-u\cdot (1+o(1))}\quad\text{where }u = \frac{\log X}{\log B}.\]
\end{theorem}
\ul{What is $o(1)$?}
\begin{definition}
    Say $f(x) = o(g(x))$ if
    \[\lim_{x\to\infty}\frac{f(x)}{g(x)} = 0\]
\end{definition}
So ``$o(1)$'' is to mean something whose limit is $0$. This is in contrast to $O(1)$ which means something whose limit is finite.

For our purposes, we say that the probability $X$ is $B$-smooth is $\approx u^{-u}$ where $u = \frac{\log X}{\log B}$.

What if we do Pollard $p-1$ for $e^{\sqrt{\log p}}$ steps? The probability of success is then \[\sqrt{\log p}^{-\sqrt{\log p}}\approx e^{-\frac{1}{2}\log\log p\cdot \sqrt{\log p}} \gg p^{-\varepsilon}\]

By brute force, doing $p^\varepsilon$ steps gives a probability of success $\approx p^{\varepsilon - 1}$.

\subsection{Quadratic Sieve}
\textbf{Goal:} find $a, b$ with $a^2\equiv b^2\pmod{N = pq}$, hence $(a-b)(a+b)\equiv 0\mod N$. $\gcd(a+b, N)$ will allow us to recover $p$ or $q$ with decent probability.

\begin{example}
    For example, if
    \begin{align*}
        (\lfloor\sqrt{N}\rfloor + 1)^2 & \equiv (\lfloor\sqrt{N}\rfloor + 1)^2 - N
        = 2^1\cdot 3^1                                                             \\
        (\lfloor\sqrt{N}\rfloor + 2)^2 & \equiv (\lfloor\sqrt{N}\rfloor + 2)^2 - N
        = (\text{not smooth})                                                      \\
        (\lfloor\sqrt{N}\rfloor + 3)^2 & \equiv (\lfloor\sqrt{N}\rfloor + 3)^2 - N
        = 2^2\cdot 3^1                                                             \\
        (\lfloor\sqrt{N}\rfloor + 4)^2 & \equiv (\lfloor\sqrt{N}\rfloor + 4)^2 - N
        = 2^1\cdot 3^2                                                             \\
                                       & \vdots
    \end{align*}
    then we have
    \[\left[(\lfloor\sqrt{N}\rfloor + 1)^3\right]^2 \equiv \left[(\lfloor\sqrt{N}\rfloor + 3) (\lfloor\sqrt{N}\rfloor + 4)\right]^2\]
    Which comes from the fact that the exponent vectors
    \[\begin{bmatrix}1\\1\end{bmatrix}, \begin{bmatrix}2\\1\end{bmatrix}, \begin{bmatrix}1\\2\end{bmatrix}\]
    are linearly dependent. We can take these mod $2$ (the parity) since we square. We can also rewrite this as\footnote{Using the definition of linear dependence}
    \[\left[(\lfloor\sqrt{N}\rfloor + 1)(\lfloor\sqrt{N}\rfloor + 3) (\lfloor\sqrt{N}\rfloor + 4)\right]^2\equiv \left(2^2\cdot 3^2\right)^2\]
\end{example}
So we can do the following steps:
\begin{enumerate}
    \item Pick smoothness bound $B$.
    \item Find integers
          \[(\lfloor \sqrt{N}\rfloor + i)^2 - N\]
          that are $B$ smooth.
    \item Find linear relationship between exponent vectors. Then we get a congruence $a^2\equiv b^2\pmod N$ after which we can try to find factors of $N$. 
\end{enumerate}
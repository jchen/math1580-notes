%!TEX root = ../notes.tex
\section{March 21, 2022}
\subsection{Elliptic Curve Elgamal}
As usual, we have some public knowledge, private key, and public key. The idea  is to replace multiplication in $\FF_p^\times$ with addition on $E$.
\begin{mdframed}
    \ul{Public Knowledge}: \\
    $p$ --- prime. \\
    $E$ --- elliptic curve over $\FF_p$. \\
    $P \in E(\FF_p)$ --- point.

    \ul{Private Key}: \\
    $n$ --- private key.

    \ul{Public Key}: \\
    $Q = n\cdot P$ --- public key.

    \ul{Encrpytion}: \\
    Bob has a message $M\in E(\FF_p)$.
    \begin{enumerate}
        \item Choose random $k$.
        \item Compute
              \begin{align*}
                  C_1 & = k\cdot P     \\
                  C_2 & = M + k\cdot Q
              \end{align*}
              Send $(C_1, C_2)$ to Alice.
    \end{enumerate}

    \ul{Decryption}: \\
    Alice will compute
    \[C_2 - n\cdot C_1 = M + k\cdot Q - nk\cdot P = M\]
\end{mdframed}

We now implement this:

\begin{lstlisting}
from ec import ext_gcd, add, minus, multiply

# Private key for Alice
n = randrange(q)

# Public key for Alice
Q = multiply(P, n, a, p)

def e(Q, M):
    """Encryption Function"""
    k = randrange(q)
    c1 = multiply(P, k, a, p)
    c2 = add(M, multiply(Q, k, a, p), a, p)
    return (c1, c2)

def d(n, C):
    """Decryption Function"""
    c1, c2 = C
    return add(c2, minus(multiply(c1, n, a, p), p), a, p)
\end{lstlisting}

The expansion factor is $2$. Even if we think of putting our message into only the $x$ coordinate, the $y$ coordinate is determined by the $x$ coordinate so the factor is still $2$.

\subsection{Elliptic Curve DSA}
Since we have Elgamal, we can also have DSA with Elliptic Curves.
\begin{mdframed}
    \ul{Public Knowledge \& Public Key}: \\
    \emph{Same as Elgamal} \\
    $p$ --- prime. \\
    $E$ --- elliptic curve over $\FF_p$. \\
    $Q = n\cdot P$ --- public key.

    \ul{Private Key}: \\
    $n$ --- private key.

    \ul{Signing}: \\
    Alice has document $d\in \ZZ/q\ZZ$.
    \begin{enumerate}
        \item Choose random $k$.
        \item Compute $(x, y) = k\cdot P$.
              \begin{align*}
                  s_1 & = x                               \\
                  s_2 & = (d + ns_1)\cdot k^{-1} \pmod{q}
              \end{align*}
    \end{enumerate}

    \ul{Verification}: \\
    We can verify the signature as follows:
    \begin{align*}
        v_1 & = d\cdot s_2^{-1} \pmod{q} \\
        v_2 & = s_1s_2^{-1} \pmod{q}
    \end{align*}
    \begin{align*}
        v_1\cdot P + v_2\cdot Q & = ds_2^{-1}\cdot P + s_1s_2^{-1} k\cdot P \\
                                & = (d + s_1 n)s_2^{-1} P                   \\
                                & = k P
    \end{align*}
    $(\text{$x$-coord of }v_1P + v_2Q) = s_1$ check to verify signature.
\end{mdframed}

Again, we can implement this: 
\begin{lstlisting}
from ec import *

def sign(n, d):
    """Signing document d with private key n"""
    k = randrange(q)
    x, _ = multiply(P, k, a, p)
    s1 = x
    s2 = ((d + n * s1) * ext_gcd(k, q)[0]) % q
    return (s1, s2)

def verify(Q, d, s):
    """Verifies document d's signature s with public key Q"""
    s1, s2 = s
    v1 = (d * ext_gcd(s2, q)[0]) % q
    v2 = (s1 * ext_gcd(s2, q)[0]) % q
    return add(multiply(P, v1, a, p), multiply(Q, v2, a, p), a, p)[0] == s1
\end{lstlisting}

\begin{remark}
    The specific numbers used in lecture is the elliptic curve used in the Bitcoin blockchain. Being able to forge signatures in this elliptic curve is to topple the Bitcoin market. 
\end{remark}